%% Generated by Sphinx.
\def\sphinxdocclass{report}
\documentclass[letterpaper,10pt,english]{sphinxmanual}
\ifdefined\pdfpxdimen
   \let\sphinxpxdimen\pdfpxdimen\else\newdimen\sphinxpxdimen
\fi \sphinxpxdimen=.75bp\relax

\usepackage[utf8]{inputenc}
\ifdefined\DeclareUnicodeCharacter
 \ifdefined\DeclareUnicodeCharacterAsOptional\else
  \DeclareUnicodeCharacter{00A0}{\nobreakspace}
\fi\fi
\usepackage{cmap}
\usepackage[T1]{fontenc}
\usepackage{amsmath,amssymb,amstext}
\usepackage{babel}
\usepackage{times}
\usepackage[Bjarne]{fncychap}
\usepackage[dontkeepoldnames]{sphinx}

\usepackage{geometry}

% Include hyperref last.
\usepackage{hyperref}
% Fix anchor placement for figures with captions.
\usepackage{hypcap}% it must be loaded after hyperref.
% Set up styles of URL: it should be placed after hyperref.
\urlstyle{same}
\addto\captionsenglish{\renewcommand{\contentsname}{Contents:}}

\addto\captionsenglish{\renewcommand{\figurename}{Fig.}}
\addto\captionsenglish{\renewcommand{\tablename}{Table}}
\addto\captionsenglish{\renewcommand{\literalblockname}{Listing}}

\addto\extrasenglish{\def\pageautorefname{page}}

\setcounter{tocdepth}{1}



\title{nbody2 Documentation}
\date{May 19, 2017}
\release{}
\author{Thomas Malthouse}
\newcommand{\sphinxlogo}{\vbox{}}
\renewcommand{\releasename}{Release}
\makeindex

\begin{document}

\maketitle
\sphinxtableofcontents
\phantomsection\label{\detokenize{index::doc}}



\chapter{vec3}
\label{\detokenize{vec3:welcome-to-nbody2-s-documentation}}\label{\detokenize{vec3::doc}}\label{\detokenize{vec3:vec3}}\index{vec3 (C type)}

\begin{fulllineitems}
\phantomsection\label{\detokenize{vec3:c.vec3}}\pysigline{\sphinxbfcode{vec3}}
\end{fulllineitems}


A 3-dimensional vector of doubles, powered by clang vector extensions. Individual elements are accessible with dot (a.x) or array (a{[}0{]}) notation.
\index{vec3\_0 (C macro)}

\begin{fulllineitems}
\phantomsection\label{\detokenize{vec3:c.vec3_0}}\pysigline{\sphinxbfcode{vec3\_0}}
\end{fulllineitems}


A zero-vector for the vec3 type
\index{vec3\_I (C macro)}

\begin{fulllineitems}
\phantomsection\label{\detokenize{vec3:c.vec3_I}}\pysigline{\sphinxbfcode{vec3\_I}}
\end{fulllineitems}


A vec3 v where v.x=1, and v.y=0, v.z=0
\index{vec3\_J (C macro)}

\begin{fulllineitems}
\phantomsection\label{\detokenize{vec3:c.vec3_J}}\pysigline{\sphinxbfcode{vec3\_J}}
\end{fulllineitems}


A vec3 v where v.x=0, v.y=1, v.z=0
\index{vec3\_K (C macro)}

\begin{fulllineitems}
\phantomsection\label{\detokenize{vec3:c.vec3_K}}\pysigline{\sphinxbfcode{vec3\_K}}
\end{fulllineitems}


A vec3 v where v.x=0, v.y=0, v.z=1
\index{vabs (C function)}

\begin{fulllineitems}
\phantomsection\label{\detokenize{vec3:c.vabs}}\pysiglinewithargsret{double \sphinxbfcode{vabs}}{{\hyperref[\detokenize{vec3:c.vec3}]{\sphinxcrossref{vec3}}}\sphinxstyleemphasis{ v}}{}
\end{fulllineitems}


This function gives the absolute value of a given vector v
\index{vec3\_eq (C function)}

\begin{fulllineitems}
\phantomsection\label{\detokenize{vec3:c.vec3_eq}}\pysiglinewithargsret{bool \sphinxbfcode{vec3\_eq}}{{\hyperref[\detokenize{vec3:c.vec3}]{\sphinxcrossref{vec3}}}\sphinxstyleemphasis{ v}, {\hyperref[\detokenize{vec3:c.vec3}]{\sphinxcrossref{vec3}}}\sphinxstyleemphasis{ w}}{}
\end{fulllineitems}


This function checks two vectors for element equality. Note that because of floating point error, this function is mostly useful for checking if a vector is uninitialized (i.e. the 0 vector.)
\index{vec3\_unit (C function)}

\begin{fulllineitems}
\phantomsection\label{\detokenize{vec3:c.vec3_unit}}\pysiglinewithargsret{{\hyperref[\detokenize{vec3:c.vec3}]{\sphinxcrossref{vec3}}} \sphinxbfcode{vec3\_unit}}{{\hyperref[\detokenize{vec3:c.vec3}]{\sphinxcrossref{vec3}}}\sphinxstyleemphasis{ v}}{}
\end{fulllineitems}


Given a vector, this function returns the unit vector (i.e. the vector pointing in the same direction with magnitude 1).


\chapter{System}
\label{\detokenize{system::doc}}\label{\detokenize{system:system}}
\sphinxhref{https://github.com/tmalthouse/nbody2/blob/master/nbody2/system.c}{system.c}
\sphinxhref{https://github.com/tmalthouse/nbody2/blob/master/nbody2/system.h}{system.h}
\index{System (C type)}

\begin{fulllineitems}
\phantomsection\label{\detokenize{system:c.System}}\pysigline{\sphinxbfcode{System}}
\end{fulllineitems}


A \sphinxcode{System} object represents a collection of celestial bodies and their state at a given time.
\index{System.bodies (C member)}

\begin{fulllineitems}
\phantomsection\label{\detokenize{system:c.System.bodies}}\pysigline{{\hyperref[\detokenize{body:c.Body}]{\sphinxcrossref{Body}}}* \sphinxbfcode{System.bodies}}
\end{fulllineitems}


An array of \sphinxcode{Body} objects
\index{System.count (C member)}

\begin{fulllineitems}
\phantomsection\label{\detokenize{system:c.System.count}}\pysigline{uint \sphinxbfcode{System.count}}
\end{fulllineitems}


The number of bodies in System.bodies
\index{time (C member)}

\begin{fulllineitems}
\phantomsection\label{\detokenize{system:c.time}}\pysigline{uint64\_t \sphinxbfcode{time}}
\end{fulllineitems}


The current time of the system, in the number of seconds since some epoch (the beginning of the simulation, perhaps?)
\index{tree (C member)}

\begin{fulllineitems}
\phantomsection\label{\detokenize{system:c.tree}}\pysigline{TreeNode \sphinxbfcode{tree}}
\end{fulllineitems}


The tree representing the system.
\index{update\_system (C function)}

\begin{fulllineitems}
\phantomsection\label{\detokenize{system:c.update_system}}\pysiglinewithargsret{void \sphinxbfcode{update\_system}}{{\hyperref[\detokenize{system:c.System}]{\sphinxcrossref{System}}}\sphinxstyleemphasis{ *sys}}{}
\end{fulllineitems}


This function updates the given system by the smallest timestep possible.


\chapter{Body}
\label{\detokenize{body:body}}\label{\detokenize{body::doc}}
\sphinxhref{https://github.com/tmalthouse/nbody2/blob/master/nbody2/body.c}{body.c}
\sphinxhref{https://github.com/tmalthouse/nbody2/blob/master/nbody2/body.h}{body.h}
\index{Body (C type)}

\begin{fulllineitems}
\phantomsection\label{\detokenize{body:c.Body}}\pysigline{\sphinxbfcode{Body}}
\end{fulllineitems}


A \sphinxcode{Body} object represents a celestial body (or other abstract celestial object.) It contains state information, body type, and other pertinent data.
\index{Body.id (C member)}

\begin{fulllineitems}
\phantomsection\label{\detokenize{body:c.Body.id}}\pysigline{uint32\_t \sphinxbfcode{Body.id}}
\end{fulllineitems}


This is a unique 32-bit integer, used to check whether two bodies are the same. Duplicate IDs will lead to inaccurate calculations. By default, IDs are based on the body’s position in the main Body array.
\index{Body.pos (C member)}

\begin{fulllineitems}
\phantomsection\label{\detokenize{body:c.Body.pos}}\pysigline{{\hyperref[\detokenize{vec3:c.vec3}]{\sphinxcrossref{vec3}}} \sphinxbfcode{Body.pos}}
\end{fulllineitems}


This holds the position information
\index{Body.vel (C member)}

\begin{fulllineitems}
\phantomsection\label{\detokenize{body:c.Body.vel}}\pysigline{{\hyperref[\detokenize{vec3:c.vec3}]{\sphinxcrossref{vec3}}} \sphinxbfcode{Body.vel}}
\end{fulllineitems}


This holds the velocity information
\index{Body.type (C member)}

\begin{fulllineitems}
\phantomsection\label{\detokenize{body:c.Body.type}}\pysigline{BodyType \sphinxbfcode{Body.type}}
\end{fulllineitems}


This holds information about the class of the body (gas, dust, DM, etc.) Not used at the moment, it will eventually be used for hydrodynamic calculation
\index{Body.mass (C member)}

\begin{fulllineitems}
\phantomsection\label{\detokenize{body:c.Body.mass}}\pysigline{double \sphinxbfcode{Body.mass}}
\end{fulllineitems}


This field ONLY exists if the macro UNIT\_MASS is undefined. Its purpose should be self-explanatory.
\index{acc (C member)}

\begin{fulllineitems}
\phantomsection\label{\detokenize{body:c.acc}}\pysigline{{\hyperref[\detokenize{vec3:c.vec3}]{\sphinxcrossref{vec3}}} \sphinxbfcode{acc}}
\end{fulllineitems}


This field holds the acceleration at the last timestep. It shouldn’t be useful for anything other than calculating the new timestep.
\index{tstep (C member)}

\begin{fulllineitems}
\phantomsection\label{\detokenize{body:c.tstep}}\pysigline{uint64\_t \sphinxbfcode{tstep}}
\end{fulllineitems}


This is the current timestep, in seconds. It will always be a power of 2.
\index{update\_timestep (C function)}

\begin{fulllineitems}
\phantomsection\label{\detokenize{body:c.update_timestep}}\pysiglinewithargsret{uint64\_t \sphinxbfcode{update\_timestep}}{{\hyperref[\detokenize{body:c.Body}]{\sphinxcrossref{Body}}}\sphinxstyleemphasis{ *b}, uint64\_t\sphinxstyleemphasis{ cur\_time}}{}
\end{fulllineitems}


This function updates the given body’s timestep, using the last step’s acceleration as a guide. The algorithm used comes from Noah Muldavin’s thesis (2013).
\index{update\_body (C function)}

\begin{fulllineitems}
\phantomsection\label{\detokenize{body:c.update_body}}\pysiglinewithargsret{void \sphinxbfcode{update\_body}}{{\hyperref[\detokenize{body:c.Body}]{\sphinxcrossref{Body}}}\sphinxstyleemphasis{ *b}, TreeNode\sphinxstyleemphasis{ *tree}}{}
\end{fulllineitems}


This function updates the given body by one timestep. The rest of the system is contained in the \sphinxcode{tree} parameter, which makes the calculations much faster than going through the bodies individually.


\section{Private Functions and Types}
\label{\detokenize{body:private-functions-and-types}}
The following entries are only visible from the \sphinxhref{https://github.com/tmalthouse/nbody2/blob/master/nbody2/body.c}{body.c} file.
They are just documented here for reference.
\index{NodeList (C type)}

\begin{fulllineitems}
\phantomsection\label{\detokenize{body:c.NodeList}}\pysigline{\sphinxbfcode{NodeList}}
\end{fulllineitems}


This type represents a simple dynamic array of TreeNode*s (similar to C++’s vector\textless{}T\textgreater{}.)
\index{NodeList\_append (C function)}

\begin{fulllineitems}
\phantomsection\label{\detokenize{body:c.NodeList_append}}\pysiglinewithargsret{void \sphinxbfcode{NodeList\_append}}{{\hyperref[\detokenize{body:c.NodeList}]{\sphinxcrossref{NodeList}}}\sphinxstyleemphasis{ *l}, TreeNode\sphinxstyleemphasis{ *n}}{}
\end{fulllineitems}


This function adds element \sphinxcode{n} to the end of NodeList \sphinxcode{l} (allocating more memory if necessary.)
\index{t\_ideal (C function)}

\begin{fulllineitems}
\phantomsection\label{\detokenize{body:c.t_ideal}}\pysiglinewithargsret{uint64\_t \sphinxbfcode{t\_ideal}}{{\hyperref[\detokenize{body:c.Body}]{\sphinxcrossref{Body}}}\sphinxstyleemphasis{ *b}}{}
\end{fulllineitems}


This function calculates the ideal timestep for a given body, using an algorithm from Muldavin 2013.
\index{node\_finder (C function)}

\begin{fulllineitems}
\phantomsection\label{\detokenize{body:c.node_finder}}\pysiglinewithargsret{\sphinxbfcode{node\_finder}}{{\hyperref[\detokenize{body:c.NodeList}]{\sphinxcrossref{NodeList}}}\sphinxstyleemphasis{ *l}, {\hyperref[\detokenize{vec3:c.vec3}]{\sphinxcrossref{vec3}}}\sphinxstyleemphasis{ pos}, TreeNode\sphinxstyleemphasis{ *tree}}{}
\end{fulllineitems}


This function walks through the given tree, adding any node that needs to be accounted for to NodeList \sphinxcode{l}.
\index{body\_acc (C function)}

\begin{fulllineitems}
\phantomsection\label{\detokenize{body:c.body_acc}}\pysiglinewithargsret{\sphinxbfcode{body\_acc}}{TreeNode\sphinxstyleemphasis{ **nodes}, size\_t\sphinxstyleemphasis{ node\_count}, {\hyperref[\detokenize{body:c.Body}]{\sphinxcrossref{Body}}}\sphinxstyleemphasis{ *b}}{}
\end{fulllineitems}


This function calculates the acceleration on Body \sphinxcode{b}, using the list of nodes found by \sphinxcode{node\_finder}.
\index{g\_acc (C function)}

\begin{fulllineitems}
\phantomsection\label{\detokenize{body:c.g_acc}}\pysiglinewithargsret{{\hyperref[\detokenize{vec3:c.vec3}]{\sphinxcrossref{vec3}}} \sphinxbfcode{g\_acc}}{{\hyperref[\detokenize{vec3:c.vec3}]{\sphinxcrossref{vec3}}}\sphinxstyleemphasis{ pos1}, {\hyperref[\detokenize{vec3:c.vec3}]{\sphinxcrossref{vec3}}}\sphinxstyleemphasis{ pos2}, double\sphinxstyleemphasis{ m2}}{}
\end{fulllineitems}


This function calculates the acceleration at \sphinxcode{pos1} caused by an object at \sphinxcode{pos2} with mass \sphinxcode{m2}, using the standard equation

r = pos2-pos1

f\_\{acc\} = frac\{G m\_2\}\{(\textbar{}r\textbar{})\textasciicircum{}2\}hat\{r\}
\index{body\_acc (C function)}

\begin{fulllineitems}
\pysiglinewithargsret{{\hyperref[\detokenize{vec3:c.vec3}]{\sphinxcrossref{vec3}}} \sphinxbfcode{body\_acc}}{TreeNode\sphinxstyleemphasis{ **nodes}, size\_t\sphinxstyleemphasis{ node\_count}, {\hyperref[\detokenize{body:c.Body}]{\sphinxcrossref{Body}}}\sphinxstyleemphasis{ *b}}{}
\end{fulllineitems}


This function calculates the total acceleration on body \sphinxcode{b}, calling \sphinxcode{g\_acc} on each of the provided nodes and adding all the results together.


\chapter{TreeNode}
\label{\detokenize{tree:treenode}}\label{\detokenize{tree::doc}}
\sphinxhref{https://github.com/tmalthouse/nbody2/blob/master/nbody2/tree.c}{tree.c}
\sphinxhref{https://github.com/tmalthouse/nbody2/blob/master/nbody2/tree.h}{tree.h}


\chapter{Indices and tables}
\label{\detokenize{index:indices-and-tables}}\begin{itemize}
\item {} 
\DUrole{xref,std,std-ref}{genindex}

\item {} 
\DUrole{xref,std,std-ref}{modindex}

\item {} 
\DUrole{xref,std,std-ref}{search}

\end{itemize}



\renewcommand{\indexname}{Index}
\printindex
\end{document}